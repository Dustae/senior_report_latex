
\documentclass[12pt,oneside,a4paper]{article}
\usepackage{polyglossia}
\setdefaultlanguage{thai}
\setotherlanguage{english}
\newfontfamily\thaifont[Script=Thai,Scale=1.23]{TH Sarabun New}
\newfontfamily\thaifonttt[Script=Thai,Scale=1.23]{TH Sarabun New} % การตั้งค่าฟอนต์สำหรับโมโนสเปซ
\defaultfontfeatures{Mapping=tex-text,Scale=1.23,LetterSpace=0.0}
\setmainfont[Scale=1.23,LetterSpace=0,WordSpace=1.0,FakeStretch=1.0,Mapping=tex-text]{TH Sarabun New}

\begin{document}

\section*{หัวข้อโครงงานวิศวกรรมคอมพิวเตอร์}
\textbf{หลักสูตรวิศวกรรมศาสตร์บัณฑิต สาขาวิศวกรรมคอมพิวเตอร์} \\
ปีการศึกษา 2567

\subsection*{Topic}
เว็บไซต์รวบรวมข่าวสารการรับสมัครนักศึกษาฝึกงานสำหรับนักศึกษาในภาควิชาวิศวกรรมคอมพิวเตอร์ \\
(Internship Hub for Computer Engineering Students)

\subsection*{Members}
\begin{itemize}
    \item นาย เจษฎา ไชยณรงค์ (640705012) \\ Email: \texttt{jedsada.jch@gmail.com}
    \item นาย ธนธรณ์ ภู่กันงาม (64070501022) \\ Email: \texttt{tanaton.phuk@gmail.com}
    \item นาย พงศ์ประวีณ์ รัตนศรี (64070501034) \\ Email: \texttt{phongprawi.ratt@gmail.com}
\end{itemize}

\subsection*{Adviser}
ดร.ทวีชัย นันทวิสุทธิวงศ์

\newpage
\section{บทที่ 1 บทนำ}
\subsection{1.1 คำสำคัญ}
Internship, Data Analyst, Web application, Web scraping

\subsection{1.2 ที่มาและความสำคัญของปัญหา}
ในปัจจุบัน การค้นหาสถานที่ฝึกงานถือเป็นขั้นตอนที่มีความสำคัญอย่างยิ่งสำหรับนักศึกษา โดยเฉพาะนักศึกษาที่กำลังเตรียมตัวเข้าสู่ตลาดแรงงานในสาขาวิศวกรรมคอมพิวเตอร์ เนื่องจากการฝึกงานไม่เพียงเปิดโอกาสให้นักศึกษาได้พัฒนาทักษะและความสามารถเท่านั้น แต่ยังช่วยให้นักศึกษาได้รับประสบการณ์การทำงานในสภาพแวดล้อมจริง ซึ่งเป็นส่วนสำคัญในการเตรียมความพร้อมสำหรับอาชีพในอนาคตหลังจากสำเร็จการศึกษา

อย่างไรก็ตาม การหาสถานที่ฝึกงานที่ตรงกับความสนใจและสาขาวิชาที่ศึกษาอยู่ยังคงเป็นความท้าทายสำคัญ ปัญหานี้เกิดจากหลายสาเหตุ เช่น นักศึกษาไม่ทราบวิธีการค้นหาบริษัทที่เปิดรับฝึกงาน หรือขาดข้อมูลและประสบการณ์ในการค้นหา อีกทั้งยังขาดช่องทางในการติดตามข่าวสารจากบริษัทที่เกี่ยวข้องกับสายงานที่ตนเองต้องการ นอกจากนี้ ยังไม่มีระบบการรวบรวมข่าวสารที่เป็นหลักเป็นแหล่งเพียงพอ ซึ่งทำให้นักศึกษาหาบริษัทที่ตรงกับความสามารถและความสนใจได้ยาก

ดังนั้น โครงงานนี้จึงมีเป้าหมายในการพัฒนาเว็บไซต์ที่สามารถช่วยนักศึกษาหาตำแหน่งฝึกงานที่เหมาะสม โดยการใช้เทคโนโลยีการดึงข้อมูลจากเว็บ (Web Scraping) จากแหล่งต่าง ๆ เช่น ประกาศรับสมัครงานหรือโพสต์รับสมัครฝึกงานจากบริษัทต่าง ๆ และนำมาจัดเก็บในรูปแบบของบล็อกที่สามารถค้นหาได้สะดวก นอกจากนี้ คณะผู้จัดทำยังนำ Machine Learning มาใช้ในการวิเคราะห์และจับคู่โปรไฟล์นักศึกษากับตำแหน่งฝึกงานที่เหมาะสม โดยอ้างอิงจากโมเดล "ประเมินสายงาน" จากโครงงานนักศึกษาปีที่ผ่านมา

นอกเหนือจากการจับคู่ตำแหน่งแล้ว ระบบยังสามารถวิเคราะห์ข้อมูลเพิ่มเติม เช่น ช่วงเวลาที่แต่ละสายงานเปิดรับสมัครฝึกงาน เพื่อช่วยลดเวลาที่นักศึกษาใช้ในการค้นหาข้อมูลและทำให้การสมัครฝึกงานมีประสิทธิภาพมากยิ่งขึ้น โครงงานนี้มุ่งหวังว่าการพัฒนาเว็บไซต์นี้จะช่วยแก้ไขปัญหาการหาสถานที่ฝึกงานและเพิ่มโอกาสให้นักศึกษาสามารถค้นหาตำแหน่งที่สอดคล้องกับทักษะและความสนใจของตนเองได้อย่างมีประสิทธิภาพ


\subsection{1.3 ประเภทของโครงงาน}
\begin{itemize}
    \item Web Application
    \item Data Visualization
\end{itemize}

\subsection{1.4 วิธีการที่นำเสนอ}
\begin{enumerate}
    \item ศึกษาและหาข้อมูลเกี่ยวกับ web scraping และเลือกเครื่องมือที่เหมาะสมกับงาน
    \item ตรวจสอบและประเมิน Machine Learning จาก COMPATH
    \item สำรวจความต้องการในการใช้งานของตัวผู้ใช้ และระบุฟังก์ชันที่ต้องการพัฒนา
    \item ออกแบบโครงสร้างของเว็บแอปพลิเคชันและฐานข้อมูล
    \item ออกแบบ User Interface
    \item พัฒนาแอปพลิเคชัน
    \item ทดสอบการใช้งานเว็บแอปพลิเคชันและปรับปรุงคุณภาพ
\end{enumerate}

\subsection{1.5 จุดประสงค์ของโครงงาน}
\begin{itemize}
    \item สร้างเว็บไซต์รวบรวมข่าวสารเกี่ยวกับการรับสมัครนักศึกษาฝึกงาน โดยเป็นข้อมูลที่อัปเดตล่าสุด
    \item ใช้ Machine Learning มาช่วยให้นักศึกษาสามารถค้นหาบริษัทที่เหมาะสมกับสายงานของตัวเองได้ และค้นหาบริษัทหรือประกาศรับฝึกงานได้
\end{itemize}

\subsection{1.6 ขอบเขตของโครงงาน}
\begin{itemize}
    \item กลุ่มเป้าหมายหลักคือ นักศึกษาชั้นปีที่ 3 ภาควิชาวิศวกรรมคอมพิวเตอร์
    \item มุ่งเน้นการใช้งานบน personal computer เป็นหลัก
\end{itemize}

\newpage
\section{บทที่ 2 ทฤษฎีและงานวิจัยที่เกี่ยวข้อง}
\subsection{2.1 ทฤษฎีและงานวิจัยที่เกี่ยวข้อง}
\subsubsection{2.1.1 Web scraping}
Web scraping เป็นกระบวนการสกัดข้อมูลออกจากเว็บไซต์ เพื่อนำข้อมูลบนเว็บไซต์ไปใช้ในการทำงาน ...

\subsubsection{2.1.2 Confusion Matrix}
Confusion Matrix เป็นหนึ่งในวิธีการประเมินประสิทธิภาพของโมเดล 

\subsection{2.2 ภาษาคอมพิวเตอร์และเทคโนโลยี}
\subsubsection{2.2.1 Visual Studio Code}
Visual Studio Code เป็นโปรแกรมแก้ไขชุดคำสั่งที่ช่วยนักพัฒนาซอฟต์แวร์ในการเขียนโค้ดได้อย่างมีประสิทธิภาพ ...

\subsubsection{2.2.2 Figma}
Figma เป็นเครื่องมือสำหรับการออกแบบ User Interface และสามารถทำงานร่วมกันแบบเรียลไทม์ ...

\subsection{2.3 งานวิจัยและเว็บไซต์ที่เกี่ยวข้อง}
\subsubsection{2.3.1 LinkedIn}
LinkedIn เป็นแพลตฟอร์มเครือข่ายสังคมออนไลน์ที่เน้นการเชื่อมโยงผู้ใช้งานที่สนใจทางด้านอาชีพ ...

\subsubsection{2.3.2 เด็กฝึกงาน.com}
เด็กฝึกงาน.com เป็นเว็บไซต์ที่ช่วยเชื่อมโยงนักศึกษาไทยกับบริษัทที่มีการเปิดรับนักศึกษาเข้าฝึกงาน ...
\end{document}
